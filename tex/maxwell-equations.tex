% \documentclass{article}
% \usepackage{graphicx} % Required for inserting images
% \usepackage{amsmath}

% \begin{document}



1. Gauss's Law

\[
\oint \vec{E} \cdot d\vec{A} = \frac{q_{\text{enc}}}{\epsilon_0}
\]

2. Gauss's Law for Magnetism

\[
\oint \vec{B} \cdot d\vec{A} = 0
\]

3. Ampere's Law

\[
\oint \vec{B} \cdot d\vec{l} = \mu_0 I_{\text{enc}} + \mu_0 \epsilon_0 \frac{d \phi_E}{dt}
\]

4. Faraday's Law

\[
\oint (\vec{E} + \vec{v} \times \vec{B}) \cdot d\vec{l} = -\frac{d \phi_B}{dt}
\]

\vspace{1em}

Assume Free Space (no charges or currents)

\vspace{1em}

1. Gauss's Law

\[
\oint \vec{E} \cdot d\vec{A} = 0
\]

2. Gauss's Law for Magnetism

\[
\oint \vec{B} \cdot d\vec{A} = 0
\]

3. Ampere's Law

\[
\oint \vec{B} \cdot d\vec{l} = \mu_0 \epsilon_0 \frac{d \phi_E}{dt}
\]

4. Faraday's Law

\[
\oint \vec{E} \cdot d\vec{l} = -\frac{d \phi_B}{dt}
\]

\vspace{1em}

Transform with Stokes' Theorem

\[
\int_C \vec{F} \cdot d\vec{l} = \iint_A (\nabla \times \vec{F}) \cdot d\vec{A}
\]

and the Divergence Theorem

\[
\iint_A \vec{F} \cdot d\vec{A} = \iiint_V (\nabla \cdot \vec{F}) \cdot dV
\]

(note each integral is closed across the entire curve/surface/volume)

\vspace{1em}

1. 

\[
\oint \vec{E} \cdot d\vec{A} = 0
\]

\[
\iiint_V \nabla \cdot \vec{E} \, dV = 0
\]

\[
\nabla \cdot \vec{E} = 0
\]

\vspace{1em}

2.

\[
\oint \vec{B} \cdot d\vec{A} = 0
\]

\[
\iiint_V \nabla \cdot \vec{B} \, dV = 0
\]

\[
\nabla \cdot \vec{B} = 0
\]

\vspace{1em}

3.

\[
\oint \vec{B} \cdot d\vec{l} = \mu_0 \epsilon_0 \frac{d \phi_E}{dt}
\]

\[
\iint_A (\nabla \times \vec{B}) \cdot d\vec{A} = \mu_0 \epsilon_0 \frac{d}{dt} \int \vec{E} \cdot d\vec{A}
\]

\[
\iint_A (\nabla \times \vec{B}) \cdot d\vec{A} = \int (\mu_0 \epsilon_0 \frac{d}{dt} \vec{E}) \cdot d\vec{A}
\]

\[
\nabla \times \vec{B} = \mu_0 \epsilon_0 \frac{d \vec{E}}{dt}
\]

\vspace{1em}

4. 

\[
\oint \vec{E} \cdot d\vec{l} = -\frac{d \phi_B}{dt}
\]

\[
\iint_A (\nabla \times \vec{E}) \cdot d\vec{A} = -\frac{d \phi_B}{dt}
\]

\[
\iint_A (\nabla \times \vec{E}) \cdot d\vec{A} = -\frac{d}{dt} \int \vec{B} \cdot d\vec{A}
\]

\[
\iint_A (\nabla \times \vec{E}) \cdot d\vec{A} = \int -\frac{d \vec{B}}{dt} \cdot d\vec{A}
\]

\[
\nabla \times \vec{E} = -\frac{d \vec{B}}{dt}
\]

\vspace{1em}

Maxwell's Equations in Differential Form:

\vspace{1em}

1. Gauss's Law

\[
\nabla \cdot \vec{E} = 0
\]

2. Gauss's Law for Magnetism

\[
\nabla \cdot \vec{B} = 0
\]

3. Ampere's Law

\[
\nabla \times \vec{B} = \mu_0 \epsilon_0 \frac{d \vec{E}}{dt}
\]

4. Faraday's Law

\[
\nabla \times \vec{E} = -\frac{d \vec{B}}{dt}
\]

\vspace{1em}

Take the curl of Faraday's Law:

\vspace{1em}

\[
\nabla \times (\nabla \times \vec{E}) = \nabla \times \left(-\frac{d \vec{B}}{dt}\right)
\]

\[
\nabla \times (\nabla \times \vec{E}) = -\frac{d}{dt} (\nabla \times \vec{B})
\]

\vspace{1em}

Substitute in Ampere's Law:

\vspace{1em}

\[
\nabla \times (\nabla \times \vec{E}) = -\frac{d}{dt} \left(\mu_0 \epsilon_0 \frac{d \vec{E}}{dt}\right)
\]

\vspace{1em}

Use the curl of curl identity:

\[
\nabla \times (\nabla \times \vec{F}) = \nabla (\nabla \cdot \vec{F}) - \nabla^2 \vec{F}
\]

\[
\nabla \cdot (\nabla \cdot \vec{E}) - \nabla^2 \vec{E} = -\frac{d}{dt} \left(\mu_0 \epsilon_0 \frac{d \vec{E}}{dt}\right)
\]

The first term goes away because of Gauss's Law:

\[
- \nabla^2 \vec{E} = -\mu_0 \epsilon_0 \frac{d^2 \vec{E}}{dt^2}
\]

Assume \(\vec{E}\) only goes in the x-direction:

\[
- \frac{\partial^2 \vec{E}}{\partial x^2} = -\mu_0 \epsilon_0 \frac{d^2 \vec{E}}{dt^2}
\]

\vspace{1em}

\[
\frac{\partial^2 \vec{E}}{\partial x^2} = \mu_0 \epsilon_0 \frac{d^2 \vec{E}}{dt^2}
\]



% \end{document}
